\documentclass{article}
\usepackage{graphicx} % Required for inserting images
\usepackage[T2A]{fontenc}


\title{Дизайн документ}
\author{Артем Антонов \\ Кононов Арсений  \\ Муслин Артемий \\ Прохоров Андрей \\ Соколов Максим}

\date{Ноябрь 2024}

\begin{document}

\maketitle

\tableofcontents
\newpage
\section{Введение}

\section{Концепция}

\subsection{Введение}
Игра представляет собой динамичный платформер с элементами тёмного фэнтези, погружающий игрока в мрачный мир средневековых подземелий. Основная цель игры — провести молодого рыцаря через опасные уровни, сразиться с различными врагами, собрать артефакты и в конечном итоге освободить королевство от древнего проклятия.

\subsection{Жанр и аудитория}
Платформер с акцентом на исследование пещер, замков и ближние бои. Игра рассчитана на игроков в возрасте от 12 лет, которые любят приключения, средневековые миры и динамичные сражения с врагами. Игра ориентирована на поклонников фэнтези с мрачной атмосферой, привлекающих визуальная детализация и продуманная боевка

\subsection{Основные особенности игры}
\begin{itemize}
    \item Мрачная средневековая атмосфера с элементами тёмного фэнтези.
    \item Боевая система с ближними атаками, уклонениями и возможностью улучшения оружия.
    \item Подземелья с возрастающей сложностью, ловушками и мини-боссами.
    \item Возможность собирать артефакты и использовать зелья для усиления персонажа.
    \item Примерная продолжительность: 5—7 часов прохождения.
\end{itemize}

\subsection{Описание игры}
Игрок управляет рыцарем, который проходит через подземелья, полные ловушек, врагов и загадочных артефактов. Основной игровой процесс включает:
\begin{itemize}
    \item Исследование лабиринтов с постепенным повышением сложности.
    \item Сражения с разнообразными врагами, включая скелетов, зомби и тёмных рыцарей.
    \item Сбор предметов, которые дают временные или постоянные улучшения.
    \item Преодоление ловушек, таких как стрелы, выпущенные из стен, или обвалы.
\end{itemize}
Игрок должен побеждать врагов, находить сокровища и артефакты, чтобы усилить своего персонажа и добраться до финального босса.

\subsection{Предпосылки создания}
\begin{itemize}
    \item Платформеры с уникальной атмосферой и хорошо продуманным сюжетом остаются популярны среди широкой аудитории.
    \item Элементы тёмного фэнтези и средневековья создают уникальный стиль, который привлекает поклонников жанра.
    \item Подобные проекты могут использоваться как основа для дальнейшего расширения, включая дополнения или многопользовательские режимы.
\end{itemize}

\subsection{Платформа}
Игра разрабатывается для ПК. Системные требования:
\begin{table}[h!]
\centering
\begin{tabular}{|l|l|l|}
\hline
\textbf{Компонент} & \textbf{Минимальные требования} & \textbf{Рекомендуемые требования} \\ \hline
Операционная система & Windows 10 64-bit & Windows 10 64-bit \\ \hline
Процессор & Intel Core i3 & Intel Core i5 \\ \hline
Оперативная память & 4 ГБ & 8 ГБ \\ \hline
Видеокарта & NVIDIA GeForce GTX 660 & NVIDIA GeForce GTX 1060 \\ \hline
Место на диске & 5 ГБ & 5 ГБ \\ \hline
\end{tabular}
\caption{Системные требования игры}
\label{}
\end{table}

\section{Функциональная спецификация}

\subsection{Принципы игры}

\subsubsection{Суть игрового процесса}
\subsubsection{Ход игры и сюжет}

\subsection{Физическая модель}

\subsection{Персонаж игрока}

\subsection{Элемент игры}

\subsection{"Искусственный интеллект"}

\subsection{Многопользовательский режим}

\subsection{Интерфейс пользователя}

\subsubsection{Блок-схема}
\subsubsection{Функциональное описание и управление}
\subsubsection{Объекты интерфейса пользователя}

\subsection{Графика и видео}

\subsubsection{Общее описание}
\subsubsection{Двумерная графика и анимация}
\subsubsection{Трехмерная графика и анимация}
\subsubsection{Анимационные вставки}

\subsection{Звуки и музыка}
\subsubsection{Общее описание}
\subsubsection{Звук и звуковые эффекты}
\subsubsection{Музыка}

\subsection{Описание уровней}
\subsubsection{Общее описание дизайна уровней}
\subsubsection{Диаграмма взаимного расположения уровней}
\subsubsection{График введения новых объектов}

\section{Контакты}

\newpage

\end{document}
