\documentclass{article}
\usepackage{graphicx} % Required for inserting images
\usepackage[T2A]{fontenc}


\title{Дизайн документ}
\author{Артем Антонов \\ Кононов Арсений  \\ Муслин Артемий \\ Прохоров Андрей \\ Соколов Максим}

\date{Ноябрь 2024}

\begin{document}

\maketitle

\tableofcontents
\newpage
\section{Введение}

\section{Концепция}

\subsection{Введение}
Игра представляет собой динамичный платформер с элементами тёмного фэнтези, погружающий игрока в мрачный мир средневековых подземелий. Основная цель игры — провести молодого рыцаря через опасные уровни, сразиться с различными врагами, собрать артефакты и в конечном итоге освободить королевство от древнего проклятия.

\subsection{Жанр и аудитория}
Платформер с акцентом на исследование пещер, замков и ближние бои. Игра рассчитана на игроков в возрасте от 12 лет, которые любят приключения, средневековые миры и динамичные сражения с врагами. Игра ориентирована на поклонников фэнтези с мрачной атмосферой, привлекающих визуальная детализация и продуманная боевка

\subsection{Основные особенности игры}
\begin{itemize}
    \item Мрачная средневековая атмосфера с элементами тёмного фэнтези.
    \item Боевая система с ближними атаками, уклонениями и возможностью улучшения оружия.
    \item Подземелья с возрастающей сложностью, ловушками и мини-боссами.
    \item Возможность собирать артефакты и использовать зелья для усиления персонажа.
    \item Примерная продолжительность: 5—7 часов прохождения.
\end{itemize}

\subsection{Описание игры}
Игрок управляет рыцарем, который проходит через подземелья, полные ловушек, врагов и загадочных артефактов. Основной игровой процесс включает:
\begin{itemize}
    \item Исследование лабиринтов с постепенным повышением сложности.
    \item Сражения с разнообразными врагами, включая скелетов, зомби и тёмных рыцарей.
    \item Сбор предметов, которые дают временные или постоянные улучшения.
    \item Преодоление ловушек, таких как стрелы, выпущенные из стен, или обвалы.
\end{itemize}
Игрок должен побеждать врагов, находить сокровища и артефакты, чтобы усилить своего персонажа и добраться до финального босса.

\subsection{Предпосылки создания}
\begin{itemize}
    \item Платформеры с уникальной атмосферой и хорошо продуманным сюжетом остаются популярны среди широкой аудитории.
    \item Элементы тёмного фэнтези и средневековья создают уникальный стиль, который привлекает поклонников жанра.
    \item Подобные проекты могут использоваться как основа для дальнейшего расширения, включая дополнения или многопользовательские режимы.
\end{itemize}

\subsection{Платформа}
Игра разрабатывается для ПК. Системные требования:
\begin{table}[h!]
\centering
\begin{tabular}{|l|l|l|}
\hline
\textbf{Компонент} & \textbf{Минимальные требования} & \textbf{Рекомендуемые требования} \\ \hline
Операционная система & Windows 10 64-bit & Windows 10 64-bit \\ \hline
Процессор & Intel Core i3 & Intel Core i5 \\ \hline
Оперативная память & 4 ГБ & 8 ГБ \\ \hline
Видеокарта & NVIDIA GeForce GTX 660 & NVIDIA GeForce GTX 1060 \\ \hline
Место на диске & 5 ГБ & 5 ГБ \\ \hline
\end{tabular}
\caption{Системные требования игры}
\label{}
\end{table}

\section{Функциональная спецификация}

\subsection{Принципы игры}
\subsubsection{Суть игрового процесса}
Игра предлагает игроку динамичный геймплей, где сочетаются элементы исследования, сражений и решения головоломок. Основное удовольствие заключается в возможности погружения в атмосферу мрачного фэнтези, исследовании запутанных подземелий, сражениях с разнообразными врагами и постепенном усилении персонажа. Игроки смогут наслаждаться напряжёнными боями, избегать ловушек и находить скрытые сокровища.
\subsubsection{Ход игры и сюжет}
Игра представляет собой череду уровней, каждый из которых содержит уникальные локации и испытания. Типичный игровой сеанс включает:
\begin{itemize}
    \item Исследование нового уровня: игрок ищет скрытые проходы, разбирается с ловушками и собирает бонусы и усиления для персонажа.
    \item Сражения с врагами: игрок использует оружие и навыки, чтобы побеждать врагов, таких как скелеты, зомби и мини-боссы.
    \item Прогресс через сюжет: после завершения уровня игрок получает части истории о древнем проклятии, порабощающем королевство.
\end{itemize}
Сюжет игры разворачивается вокруг молодого рыцаря, которому поручено спасти королевство от тёмного проклятия. В процессе он находит древние руины, узнаёт историю прошлого и сталкивается с финальным боссом — духом древнего лорда, заключённым в магическую броню.

\subsection{Физическая модель}

\subsubsection{Перемещения}
Игрок управляет рыцарем, который может бегать, прыгать, уклоняться от атак врагов и преодолевать препятствия. Для перемещений задействована система ускорения, позволяющая разгоняться или замедляться на разных поверхностях уровня.
\subsubsection{Боевые действия}
Система боя включает:
\begin{itemize}
    \item Ближние атаки мечом с возможностью нанесения серии ударов.
    \item Применение для нанесения урона бонусов, найденных на уровнях
    \item Уклонения и блоки для избегания урона.
    \item Особые атаки, активируемые при заполнении шкалы энергии.
\end{itemize}

\subsubsection{Физическая модель}
Физическая модель учитывает столкновения с врагами, преградами и использование окружения для защиты(например, уничтожение хрупких стен).

Для расчёта урона и взаимодействия с врагами используются следующие формулы:

\[
\textrm{Урон врага} = (\textrm{Сила оружия} + \textrm{Бонус от артефактов}) \times (1 + \textrm{Критический множитель})
\]

\[
\textrm{Урон от ловушки} = \textrm{Мощность ловушки} \times \textrm{Фактор расстояния}
\]

\[
\textrm{Урон игрока} = \textrm{Урон врага} \times (1 - \textrm{Защита игрока}) \times (1 + \textrm{Критический множитель врага})
\]

\textbf{Пример 1:} Если сила оружия равна 10, бонус от артефактов — 5, а критический множитель — 0.5, то общий урон, который игрок наносит врагу, составит:
\[
\textrm{Урон врага} = (10 + 5) \times (1 + 0.5) = 22.5
\]

\textbf{Пример 2:} Если базовый урон врага равен 20, защита игрока составляет 30\% (0.3), а критический множитель врага — 0.5, то урон, получаемый игроком, составит:
\[
\textrm{Урон игрока} = 20 \times (1 - 0.3) \times (1 + 0.5) = 21
\]
\subsection{Персонаж игрока}

Игрок управляет молодым рыцарем, экипированным базовым мечом и лёгкими доспехами. Характеристики персонажа:
\begin{itemize}
    \item \textbf{Здоровье:} Индикатор, уменьшающийся при получении урона.
    \item \textbf{Выносливость:} Расходуется при уклонениях и мощных атаках.
    \item \textbf{Энергия:} Используется для специальных способностей.
\end{itemize}
Игрок может находить артефакты, которые временно или постоянно увеличивают эти показатели.

\subsection{Элементы игры}
Оружие:
\begin{itemize}
    \item \textbf{Базовый меч}: стандартное оружие, с возможностью улучшения (увеличение урона, скорость атаки).
    \item \textbf{Молоты и копья}: редкие находки с уникальными атаками. Молоты наносят большой урон, но имеют медленную скорость атаки, а копья позволяют атаковать на расстоянии, но с меньшим уроном.
    \item \textbf{Зелья}: различные зелья, которые игрок может использовать для улучшения своих характеристик или воздействия на врагов. Некоторые зелья могут, например, отравлять врагов, нанося им постепенный урон, в то время как другие могут повышать скорость или силу атаки.
    \item \textbf{Щиты}: элементы защиты, которые могут блокировать атаки врагов. Щиты могут быть улучшены или иметь уникальные эффекты (например, поглощение магии).
\end{itemize}
![alt text](https://i.pinimg.com/originals/a4/4b/4f/a44b4f6c0213ca795c1735bc683791af.png)
![alt text] (https://img.craftpix.net/2019/09/Spear-2D-Weapon-Pack3.webp)
![alt text] (https://i.pinimg.com/736x/a1/c5/7a/a1c57a0d6363139d7ea3df448eb29fb2.jpg)
Артефакты:
\begin{itemize}
    \item \textbf{Кольца}: усиливают различные характеристики игрока, например, здоровье, выносливость или скорость.
    \item \textbf{Амулеты}: дают бонусы к боевым навыкам или обеспечивают защиту от определенных типов врагов (например, амулет, защищающий от огненных атак).
    \item \textbf{Свитки}: позволяют игроку изучать новые способности или временно усиливать свои характеристики.
\end{itemize}
![alt text] (https://i.redd.it/v9dbis1x0aj71.jpg)
![alt text] (https://mir-s3-cdn-cf.behance.net/project_modules/max_1200/ab6c90115558811.6050dde28543d.jpg)
Карты:
\begin{itemize}
    \item \textbf{Карты местности}: игрок может находить карты, которые раскрывают часть скрытых путей или указывают на расположение сокровищ и ловушек в подземелье.
    \item \textbf{Карты врагов}: показывают расположение сильных врагов или мини-боссов, которых стоит избегать или победить для получения ценного снаряжения.
    \item \textbf{Карты сундуков}: карты, которые дают подсказки о местах нахождения редких сундуков с уникальными артефактами или оружием.
\end{itemize}
![alt text] (https://i.pinimg.com/originals/14/f5/4f/14f54f392da1642ccd6de75c61514335.png)
![alt text] (https://areajugones.sport.es/wp-content/uploads/2017/03/slain-back-from-hell-10.jpg)
NPC и враги:
\begin{itemize}
    \item Враги:
    \begin{itemize}
        \item Скелеты: медлительные, но наносят большой урон.
        \item Зомби: быстрые, атакуют группами.
        \item Тёмные рыцари: элитные враги с мощным оружием.
    \end{itemize}
    \item NPC:
    \begin{itemize}
        \item Торговец: продаёт зелья и улучшения.
        \item Учёный: раскрывает детали сюжета.
    \end{itemize}
\end{itemize}
![alt text] (https://play-lh.googleusercontent.com/iDyZMK_crQ9FVX084l6mA2tfOKXECCL7YuSgkUNxfhlPsouHUtZOd7EMMF2v3hrF6Ck=w1052-h592)
![alt text] (https://i.ytimg.com/vi/5DbCX647thU/maxresdefault.jpg)
![alt text] (https://avatars.mds.yandex.net/i?id=66cc911d1203c387cf94612dd41299f0ec9ac263-10927283-images-thumbs&n=13)
![alt text] (https://sun9-79.userapi.com/impg/UKrYTPImI6fiBzSiNZcDDSQz0Ya_ucBd6Qj2eA/kcWOIdo-_NE.jpg?size=780x690&quality=95&sign=67e8cdd27f91ddd895d89a8698b9bd10&c_uniq_tag=dLYcWQlmlgfSoxyB25OuvoLj1ZSEBEOot6AfQ4yyS3Y&type=album)

\subsection{"Искусственный интеллект"}
AI врагов разработан для создания динамичных и напряжённых сражений:
\begin{itemize}
    \item Скелеты преследуют игрока, блокируя его путь.
    \item Зомби атакуют группами, пытаясь окружить.
    \item Тёмные рыцари используют тактику, например, уклонения и контратаки.
\end{itemize}
\subsection{Многопользовательский режим}
Не предусмотрен
\subsection{Интерфейс пользователя}

\subsubsection{Блок-схема}
\subsubsection{Функциональное описание и управление}
\begin{itemize}
    \item \textbf{Главное меню:}
    \begin{itemize}
        \item Кнопка "Новая игра" — начинает игру с первого уровня.
        \item Кнопка "Настройки" — открывает настройки игры (громкость, управление).
        \item Кнопка "Выход" — завершает игру.
    \end{itemize}
    \item \textbf{Экран игры:}
    \begin{itemize}
        \item \textbf{Элементы интерфейса:}
        \begin{itemize}
            \item Индикатор здоровья игрока.
            \item Текущее оружие игрока.
            \item Счетчик врагов.
        \end{itemize}
        \item \textbf{Управление:}
        \begin{itemize}
            \item Передвижение: клавиши A/D или стрелки влево/вправо.
            \item Прыжок: пробел.
            \item Атака ближнего боя: левая кнопка мыши.
            \item Использование магии (дальняя атака): правая кнопка мыши.
        \end{itemize}
    \end{itemize}
    \item \textbf{Экран окончания уровня:}
    \begin{itemize}
        \item Кнопка "Продолжить" — загружает следующий уровень.
        \item Кнопка "В меню" — возвращает в главное меню.
    \end{itemize}
    \item \textbf{Экран проигрыша:}
    \begin{itemize}
        \item Сообщение о неудаче.
        \item Кнопка "Повторить" — перезапускает уровень.
        \item Кнопка "В меню".
    \end{itemize}
\end{itemize}
\subsubsection{Объекты интерфейса пользователя}

\subsection{Графика и видео}

\subsubsection{Общее описание}
\subsubsection{Двумерная графика и анимация}
\subsubsection{Трехмерная графика и анимация}
\subsubsection{Анимационные вставки}

\subsection{Звуки и музыка}
\subsubsection{Общее описание}
\subsubsection{Звук и звуковые эффекты}
\subsubsection{Музыка}

\subsection{Описание уровней}
\subsubsection{Общее описание дизайна уровней}
\subsubsection{Диаграмма взаимного расположения уровней}
\subsubsection{График введения новых объектов}

\section{Контакты}

\newpage

\end{document}
